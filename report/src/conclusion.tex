\section{Conclusion du projet}
\label{sec:conclusion-projet}

Dans le cadre du cours d'\emph{Administration Linux}, il nous a été demandé
d'administrer un serveur sous Linux. \\

Ce projet nous a permis de mettre en pratique la théorie vue au premier
quadrimestre de manière réaliste. Nous avions libre choix quant à la distribution
ainsi qu'au type de gestion des sauvegardes, à condition de les justifier.

Le serveur devait contenir un partage NFS et Samba, un serveur Web, FTP, MySQL,
DNS ainsi que NTP, et le support du module SSH.

La mise en place de ces services nous a obligé à nous documenter et à prendre
des précautions telles que :
\begin{itemize}
\item automatiser les installations et configurations au travers de scripts;
\item documenter ces scripts;
\item sauvegarder la machine virtuelle lors de chaque changement sur le serveur;
\item contrôler le bon fonctionnement de chaque élément. \\
\end{itemize}

De plus, nous avons dû nous répartir les tâches afin d'avancer efficacement
dans le projet. \\
Dans cette optique, le passage par
GitHub\footnote{\url{https://fr.wikipedia.org/wiki/GitHub}}, un outils de
gestion de développement et de version, nous a été bénéfique. \\
Celui-ci nous permettant de toujours travailler sur la dernière version des
scripts sans rencontrer de problèmes de synchronisation. \\

Grâce à l'administration de ce serveur, nous avons pu approfondir nos
connaissances dans la configuration de services et dans l'écriture de scripts.


%%% Local Variables:
%%% mode: latex
%%% TeX-master: t
%%% End:
