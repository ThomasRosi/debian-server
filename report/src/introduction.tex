\section{Présentation générale du projet}
\label{sec:pres-gener-du}

\subsection{Introduction}
\label{subsec:introduction}

Dans le cadre de ce projet, il nous a été demandé d'administrer un serveur sous
Linux.

Le choix de la distribution ainsi que la gestion des sauvegardes est libre et
devra être justifié.

Le serveur devra contenir :
\begin{itemize}
\item un partage NFS qui permettra aux utilisateurs du réseaux local d'y stocker
des fichiers ;

\item un partage Samba qui permettra aux utilisateurs de Windows d'accéder à ce
même partage ;

\item un serveur Web, FTP, MySQL et DNS qui permettra un hébergement
multi-utilisateurs ;
  \begin{itemize}
  \item[\tiny$\bullet$] le serveur FTP permettra à chaque utilisateur d'accéder
à son dossier Web ;
  \item[\tiny$\bullet$] le serveur DNS contiendra une zone qui sera
    indispensable pour les sites Web de l'utilisateur ;

  \item[\tiny$\bullet$] le serveur DNS fera également office de DNS cache pour
le réseau local.
  \end{itemize}

\item un serveur NTP pour que les machines du réseau local puissent se
synchroniser ;

\item le support du module SSH.
\end{itemize}

\newpage

\subsection{Déontologie}
\label{subsec:déontologie}

En tant qu'administrateurs du serveur, nous serons tenus de suivre de nombreuses
règles telles que :
\begin{itemize}
\item la documentation des actions entreprises sur le serveur;
\item l'automatisation des installations et configurations au travers de scripts;
\item la sécurité : mise en place de mots de passe forts, du SSH, etc. ;
\item la vigilance et la prévoyance, par exemple par la mise en place de
  sauvegardes avant et après chaque changement sur le serveur ;
\item le contrôle du bon fonctionnement de chaque élément.
\end{itemize}

\subsection{Sécurité}
\label{subsec:securite}

Du côté de la sécurité, quelques contraintes seront prises en compte :
\begin{itemize}
\item mise en place d'une politique utilisateur ;
\item mise en place de quotas ;
\item partitionnement et gestion du disque (LVM et RAID) ;
\item mise en place d'une stratégie de sauvegarde ;
\item désactivation des éléments inutiles et des mises à jours ;
\item mise en place d'un antivirus, d'un firewall, etc.
\end{itemize}

%%% Local Variables:
%%% mode: latex
%%% TeX-master: t
%%% End:
