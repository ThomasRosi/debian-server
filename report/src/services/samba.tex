\subsection{Samba}
\label{subsec:samba}

Samba est un outil permettant de partager des dossiers et des imprimantes à
travers un réseau local.

Son utilisation est conseillée pour partager de manière
simple des ressources entre plusieurs ordinateurs.

Celui-ci est compatible avec les systèmes d'exploitation suivants : Windows,
macOS, ainsi que des systèmes GNU/Linux, *BSD et Solaris} dans lesquels une
implémentation de Samba est installée.

\subsubsection{Configuration}
\label{subsubsec:configuration}

La configuration du serveur Samba se déroule en trois parties, mais tout
d'abord, il est nécessaire de créer le dossier de partage et de lui donner les
droits appropriés.

\begin{lstlisting}[language=bash]
  mkdir -p /srv/share/users/
  chown -R root:users /srv/share/users/
  chmod -R 775 /srv/share/users/
\end{lstlisting}

La totalité de l'implémentation se trouve dans le fichier
\textit{/etc/samba/smb.conf}.

\begin{enumerate}
\item configuration de Samba (désignation du \emph{workgroup}, choix
  du nom de \emph{netbios}, etc.) ;

  \begin{lstlisting}[language=bash]
    # Installation of Samba.
    apt-get install libcups2 samba samba-common cups
    # If you don't know the name of the workgroup
    # run this command on the Windows client to get
    # the workgroup name: net config workstation.
    [global]
    workgroup = WORKGROUP
    server string = Samba Server %v
    netbios name = debian
    security = user
    map to guest = bad user
    dns proxy = no
  \end{lstlisting}

\item configuration du partage pour le groupe \og \textit{users} \fg
  (désignation du chemin, des droits, etc.) ;

  \begin{lstlisting}[language=bash]
    [users]
    comment = All Users
    path = /srv/share/users
    valid users = @users
    force group = users
    create mask = 0660
    directory mask = 0771
    writable = yes
  \end{lstlisting}

\item configuration du partage du dossier \og \textit{home} \fg des utilisateurs
  (désignation des droits, vérification de l'identité, etc.).

  \begin{lstlisting}[language=bash]
    [homes]
    comment = Home Directories
    browseable = no
    valid users = %S
    writable = yes
    create mask = 0700
    directory mask = 0700
  \end{lstlisting}
\end{enumerate}

%%% Local Variables:
%%% mode: latex
%%% TeX-master: t
%%% End:
