\subsection{NTP}
\label{subsec:ntp}

Le NTP ({\emph{Network Time Protocol}), est le protocole utilisé afin
de synchroniser les machines du réseau local en fonction de l'horloge du
serveur.

\subsubsection{Principe}
\label{subsubsec:principe}

Bien que tout équipement informatique dispose d'une horloge, celle-ci
dérive comme toute montre ordinaire, ce qui peut amener à des erreurs de
synchronisation. \\
La nécessité de synchroniser des équipements en réseau paraît alors évidente.

Chaque machine peut être à la fois serveur et cliente.  Celle-ci sélectionnera
un serveur de temps dans sa configuration, et récupérera l'heure, ainsi qu'un
numéro de strate, \emph{n}, et se déclarera de strate \emph{n + 1}.

L'architecture du réseau est en arborescence, et divisée en trois couches :
\begin{enumerate}
\item les plus précises (horloges atomiques, récepteurs GPS, ...) sont
de \emph{strate 0};

\item les serveurs diffusant l'heure des sources sont de \emph{strate 1};

\item les serveurs de \emph{strate 2} sont généralement accessibles au public.
\end{enumerate}

\subsubsection{Configuration du serveur}
\label{subsubsec:configuration-serveur}

La totalité de l'implémentation se trouve dans le fichier
\textit{/etc/ntp.conf}

Voici les différentes étapes et options qui ont été effectuées :
\begin{itemize}
\item activation des statistiques NTP;
\item ajout de trois serveurs (un belge et deux européens);
\item activation de l'échange de l'heure avec tout le monde (aucune
  modification n'est acceptée);
\item activation de la synchronisation avec les machines du réseau local.
\end{itemize}

\newpage

\begin{lstlisting}[language=bash]
...

# Adjust time server.
ntpdate 1.be.pool.ntp.org

driftfile /var/lib/ntp/ntp.drift

# Statistics loopstats peerstats clockstats.
filegen loopstats file loopstats type day enable
filegen peerstats file peerstats type day enable
filegen clockstats file clockstats type day enable

# You do need to talk to an NTP server or two (or three).
server 1.be.pool.ntp.org iburst
server 3.europe.pool.ntp.org
server 2.europe.pool.ntp.org

# By default, exchange time with everybody, but don't
# allow configuration.
restrict -4 default kod notrap nomodify nopeer noquery
restrict -6 default kod notrap nomodify nopeer noquery

# Local users may interrogate the ntp server more closely.
restrict 127.0.0.1
restrict 10.1.0.0 mask 255.255.0.0 nomodify notrap nopeer
restrict ::1

# To provide time to the local subnet.
broadcast 10.1.255.255

...
\end{lstlisting}

\underline{Remarque :} l'adresse de diffusion (\textit{broadcast}) a été adapté
en fonction du réseau.


\subsubsection{Configuration du client}
\label{subsubsec:configuration-client}

Tout comme le serveur, la totalité de l'implémentation se trouve dans le fichier
\textit{/etc/ntp.conf}

Sur le client, la configuration est beaucoup plus simple :
\begin{itemize}
    \item activation des statistiques NTP;
    \item ajout du serveur local.
\end{itemize}

\begin{lstlisting}[language=bash]
...

# Adjust time server.
ntpdate 1.be.pool.ntp.org

# File containing the average deviation.
driftfile /var/lib/ntp/ntp.drift

# Desired Statistics.
statistics loopstats peerstats clockstats
filegen loopstats file loopstats type day enable
filegen peerstats file peerstats type day enable
filegen clockstats file clockstats type day enable

# You do need to talk to an NTP server or two (or three).
server 10.1.214.184 prefer

...
\end{lstlisting}

\underline{Remarque :} l'adresse IP ``10.1.214.184'' étant celle du serveur NTP.

%%% Local Variables:
%%% mode: latex
%%% TeX-master: t
%%% End:
