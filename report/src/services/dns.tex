\subsection{DNS}
\label{subsec:dns}

Le \textbf{DNS}, ou \textit{\textbf{Domain Name System}}, est un service
permettant de résoudre un nom de domaine. \\
De fait, les serveurs étant identifiés par leur adresse IP, il a fallu créer un
processus afin d'associer leur adresse à un nom plus simple à retenir,
le \og \textit{nom de domaine} \fg.


\subsubsection{Sélection du DNS}
\label{subsubsec:selection-dns}

Nous avons choisi d'utiliser \textbf{BIND}, pour \textit{\textbf{Berkeley
Internet Name Daemon}}. \\
C'est \textit{le} serveur DNS le plus utilisé sur Internet, spécialement sur les
systèmes de type UNIX et est devenu de facto un standard.


\subsubsection{Mise en place}
\label{subsubsec:miseen-place}

Le DNS a été installé et configuré sur le serveur en différentes étapes :
\begin{itemize}

    \item[$\bullet$] installation de BIND9;
    \item[$\bullet$] création des ACL \textit{(\textbf{Access Control List})}
    définissant le réseau local;
    \item[$\bullet$] création et configuration du serveur DNS en lui-même :
    \begin{itemize}

        \item acceptation des requêtes uniquement pour le réseau interne;
        \item configuration des forwarders;
        \item activation de \textit{\textbf{DNSSEC}} qui sécurise les
        données envoyées par le DNS;
        \item activation de l'écoute des requêtes IPv6;
        \item implémentation de la
        RFC1035\footnote{\url{http://abcdrfc.free.fr/rfc-vf/rfc1035.html}}
        \footnote{\url{http://www.bortzmeyer.org/1035.html}}. \\

    \end{itemize}

\end{itemize}


%%% Local Variables:
%%% mode: latex
%%% TeX-master: t
%%% End:
