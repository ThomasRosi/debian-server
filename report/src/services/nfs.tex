\subsection{NFS}
\label{subsec:nfs}

Le \textbf{NFS}, ou \textit{\textbf{Network File System}}, est un protocole qui
permet à un ordinateur d'accéder à des fichiers distants via un réseau. \\
Ce système de fichiers en réseau permet de partager des données principalement
entre systèmes UNIX.


\subsubsection{Constatation}
\label{subsubsec:constatation}

Avant de commencer, il est à remarquer que, quelle que soit sa version, NFS est
a déployer dans un réseau local et n'a pas de vocation à être ouvert sur internet. \\
En effet, les données qui circulent sur le réseau ne sont pas chiffrées et les
droits d'accès sont accordés en fonction de l'adresse IP du client
\textit{(qui peut être usurpée)}.


\subsubsection{Configuration côté serveur}
\label{subsubsec:config-serveur}

Voici les différentes étapes et options que nous avons effectuées :
\begin{itemize}

    \item[$\bullet$] installation des différents services indispensables au NFS;
    \item[$\bullet$] création du dossier de partage, et ajout de droits
    spécifiques;
    \item[$\bullet$] activation du partage sur le réseau local et configuration
    dudit partage \textit{(autorise la lecture et l'écriture, retire les droit
    \textbf{root} à distance et et désactivation de la vérification de sous-répertoires)};
    \item[$\bullet$] mise à jour de la tables des systèmes de fichiers partagés.

\end{itemize}


\subsubsection{Configuration côté client}
\label{subsubsec:config-client}

Sur le client, la configuration est similaire :
\begin{itemize}

    \item[$\bullet$] installation des différents services indispensables au NFS;
    \item[$\bullet$] création du dossier de partage, et ajout de droits
    spécifiques;
    \item[$\bullet$] installation d'\textit{AutoFS};
    \item[$\bullet$] configuration d'AutoFS \textit{(création d'un point de
    montage lors de l'accès au répertoire, durée d'activité après le dernier
    accès au dossier partagé $\Rightarrow$ au moins 30 secondes pour un partage samba, etc.)}.

\end{itemize}


%%% Local Variables:
%%% mode: latex
%%% TeX-master: t
%%% End:
