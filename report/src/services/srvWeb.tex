\subsection{Serveur Web}
\label{subsec:serveur-web}

Pour la création du serveur Web, Apache a été utilisé.

\subsubsection{Configuration côté serveur}
\label{subsubsec:config-srv}

La totalité de l'implémentation se trouve dans les fichiers \\
\textit{/etc/apache2/mods-available/mpm\_prefork.conf} \\
\textit{/etc/apache2/mods-available/mpm\_event.conf}

Voici les différentes étapes et options qui ont été effectuées :
\begin{itemize}
\item installation de la version 2 d'Apache ;

  \begin{lstlisting}[language=bash]
    # Installation of Apache 2.
    apt-get install -y apache2 apache2-doc apache2-utils
  \end{lstlisting}

\item configuration des modules multiprocessus (MPMs) ;

  \begin{lstlisting}[language=bash]
    # prefork MPM.
    <IfModule mpm_prefork_module>
            StartServers               4
            MinSpareServers            20
            MaxSpareServers            40
            MaxRequestWorkers          200
            MaxConnectionsPerChild     4500
    </IfModule>
  \end{lstlisting}

\item désactivation du module d'évenement et activation de divers modules ;

  \begin{lstlisting}[language=bash]
    # On Debian 8, the event module is enabled by default.
    # This will need to be disabled, and the prefork
    # module enabled.
    a2dismod mpm_event
    a2enmod mpm_prefork
    a2enmod userdir
    a2enmod rewrite
  \end{lstlisting}

\item configuration du module d'évenement ;

  \begin{lstlisting}[language=bash]
    # event MPM.
    <IfModule mpm_event_module>
            StartServers               2
            MinSpareServers            25
            MaxSpareServers            75
            ThreadLimit                64
            ThreadsPerChild            25
            MaxRequestWorkers          150
            MaxConnectionsPerChild     3000
    </IfModule>
  \end{lstlisting}

\item configuration du serveur Apache ;

  \begin{lstlisting}[language=bash]
    # Disable the default Apache virtual host.
    a2dissite 000-default.conf

    # Creation of the Web folder.
    mkdir -p /srv/www/example/www
  \end{lstlisting}

\item masquage de la version de PHP pour les utilisateurs ;

  \begin{lstlisting}[language=bash]
    # Hide the version of PHP for users.
    min_info
  \end{lstlisting}

\item installation du support PHP.

  \begin{lstlisting}[language=bash]
    # Install PHP support.
    apt-get install -y php5 php-pear
  \end{lstlisting}

\underline{Remarque :} les dossiers des sites Web ainsi que des logs sont créés
lors de l'ajout d'un utilisateur.
\end{itemize}

%%% Local Variables:
%%% mode: latex
%%% TeX-master: t
%%% End:
