\subsection{iptables}
\label{subsec:iptables}

iptables permet de configurer les règles du pare-feu en IPv4.

\subsubsection{Écriture du script}
\label{subsubsec:ecriture-script}

Le script est \textit{/bin/script/iptables/iptables-conf.sh}.

Voici comment il est structuré :
\begin{itemize}
    \item \textit{suppression des règles par défaut}

        \begin{lstlisting}[language=bash]
            # Flushing all rules from all tables.
            iptables -F ; iptables -X
            iptables -t nat -F ; iptables -t nat -X
            iptables -t mangle -F ; iptables -t mangle -X
            iptables -t raw -F ; iptables -t raw -X
        \end{lstlisting}

    \item mise en place de règles par défaut

        \begin{lstlisting}[language=bash]
            # Setting default filter policy.
            iptables -P INPUT DROP
            iptables -P OUTPUT DROP
            iptables -P FORWARD DROP

            # Allow loopback access.
            iptables -A INPUT -i lo -j ACCEPT
            iptables -A OUTPUT -o lo -j ACCEPT
        \end{lstlisting}

    \item acceptation des différents services tels que les \emph{ping}, le
    \emph{DNS}, \emph{HTTP(S)}, \emph{NTP}, \emph{NFS}, \emph{Samba} et \emph{SSH}.
    Voici la configuration du DNS :

        \begin{lstlisting}[language=bash]
            # Allow DNS (53)
            iptables -A OUTPUT -p udp -o eth0 --dport 53 -j ACCEPT
            iptables -A INPUT -p udp -i eth0 --sport 53 -j ACCEPT
        \end{lstlisting}

\end{itemize}

%%% Local Variables:
%%% mode: latex
%%% TeX-master: t
%%% End:
