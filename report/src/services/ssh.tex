\subsection{SSH}
\label{subsec:ssh}

Le \emph{SSH} (\emph{Secure Shell}), est un protocole de communication
sécurisé. Il impose un échange de clés de chiffrement en début de connexion.

\subsubsection{Type d'authentification}
\label{subsubsec:type-authentification}

Il existe plusieurs façons de s'authentifier sur le serveur via SSH.

Les deux plus utilisées sont :

\begin{enumerate}
\item L'authentification par mot de passe ;
\item L'authentification par clés publique et privée du client.
\end{enumerate}

L'identification automatique par clés a été mise en place pour ce serveur. De ce
fait, il est nul nécessaire d'entrer le mot de passe à chaque connexion à
distance. \\ Cette méthode est plus complexe à mettre en place, mais elle
surtout plus pratique.

On remarque rapidement son utilité si on se connecte fréquemment au serveur, car
plus aucun mot de passe n'est demandé.

\subsubsection{Implémentation}
\label{subsubsec:implementation}

Tout d'abord, le serveur a été configuré respectant ces critères :

\begin{itemize}
    \item installation de \textit{openssh} ;
    \item changement de port et passage à la version 2 de SSH pour plus de
    sécurité ;
    \item ajout d'une bannière ;
    \item de la connexion en tant que \textbf{root} ;
    \item déconnexion après 120 secondes d'inactivité ;
    \item désactivation de la connexion par mot de passe, vu que l'authentification
    passe par les clés RSA.
\end{itemize}

Ensuite, une génération de un chifrement d'une paire de clés publique / privée
sur la machine client a été nécessaire. \\ Une fois cela fait, la clé publique a
été enregistrée sur le serveur afin de l'accepter dans le futur.


%%% Local Variables:
%%% mode: latex
%%% TeX-master: t
%%% End:
