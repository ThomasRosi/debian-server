\subsection{SELinux}
\label{subsec:selinux}

SELinux (\emph{Security-Enhanced Linux}) permet de définir des politiques
d'accès à différents éléments du système d'exploitation. Ces éléments peuvent
être des processus (\emph{démons}), ou encore des fichiers.

Dans le cadre de la mise en place d'un serveur conséquent, il aurait fallu
implémenter ce type de service.

Pour ce projet, il a été fait abstraction de ce service.

%%% Local Variables:
%%% mode: latex
%%% TeX-master: t
%%% End:
