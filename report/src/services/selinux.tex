\subsection{SELinux}
\label{subsec:selinux}

SELinux (\emph{Security-Enhanced Linux}) permet de définir des politiques
d'accès à différents éléments du système d'exploitation. Ces éléments peuvent
être des processus (\emph{démons}), ou encore des fichiers.

Dans le cadre de la mise en place d'un serveur conséquent, il aurait fallu
implémenter ce type de service.

Pour ce projet, un début d'implémentation a été fait, mais vu que l'ensemble des
services n'était pas encore répertorié, il a été décidé de faire abstraction de
ce service afin d'éviter que certains services se retrouvent bloqués.

  \begin{lstlisting}[language=bash]
    # Installation of SELinux.
    apt-get install -y selinux-basics selinux-policy-default
                       auditd

    # Configure GRUB and PAM and create /.autorelabel
    selinux-activate
    # Reboot the system.
    reboot
    # Check that everything has been setup correctly and
    # catch common SELinux problems.
    check-selinux-installation
    # You can see all would-be denials since the last reboot.
    audit2why -al
    # SELinux, enable enforcing mode temporarily by running:
    setenforce 1
    # To enable enforcing mode permanently, you need to add it to
    # the kernel command.
    echo "enforcing=1" >> /etc/default/grub
    # Reboot the system.
    reboot
    # Some of the PAM config files need to have "session required
    # pam_selinux.so multiple"
  \end{lstlisting}

%%% Local Variables:
%%% mode: latex
%%% TeX-master: t
%%% End:
