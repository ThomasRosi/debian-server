\section{Samba}
\label{sec:samba}

\textbf{Samba} est un outil permettant de partager des dossiers et des
imprimantes à travers un réseau local. Son utilisation est conseillée pour
partager de manière simple des ressources entre plusieurs ordinateurs  \\
Il est compatible avec les systèmes d'exploitation suivants :
\textit{\textbf{Windows}}, \textit{\textbf{macOS}}, ainsi que des systèmes \textit{\textbf{GNU/Linux}},
\textit{\textbf{*BSD}} et \textit{\textbf{Solaris}} dans lesquels une implémentation de Samba est
installée.


\subsection{Configuration}
\label{subsec:configuration}

La configuration du serveur Samba se déroule en trois parties, mais tout d'abord,
il faut créer le dossier de partage et lui donner les droits appropriés. \\

\begin{enumerate}

    \item configuration de Samba \textit{(désignation du \textbf{workgroup},
    choix du nom de \textbf{netbios}, etc.)};
    \item configuration du partage pour le groupe \og \textit{users} \fg
    \textit{(désignation du chemin, des droits, etc.)};
    \item configuration du partage du dossier \og \textit{home} \fg des utilisateurs
    \textit{(désignation des droits, vérification de l'identité, etc.)};

\end{enumerate}

%%% Local Variables:
%%% mode: latex
%%% TeX-master: t
%%% End:
