\section{NTP}
\label{sec:ntp}

Le \textbf{NTP}, ou \textit{\textbf{Network Time Protocol}}, est le protocole que nous
avons utilisé afin de synchroniser les machines du réseau local avec l'horloge
du serveur.


\subsection{Principe}
\label{subsec:principe}

En effet, bien que tout équipement informatique dispose d'une horloge, celle-ci
dérive comme toute montre ordinaire, ce qui peut amener des erreurs de
synchronisation. \\
La nécessité de synchroniser des équipements en réseau paraît alors évidente. \\

Chaque machine peut être à la fois serveur et client.\\
Elle sélectionnera un serveur de temps dans sa configuration, et récupérera
l'heure, ainsi qu'un numéro de strate, \textit{\textbf{n}}, et se déclarera
de strate \textit{\textbf{n+1}}.\\

L'architecture du réseau est en arborescence, et divisée en trois couches :
\begin{enumerate}

    \item les sources les plus précises \textit{(horloges atomiques,
    récepteurs GPS…)} sont de \textbf{strate 0};
    \item les serveurs diffusant l'heure des sources sont de \textbf{strate 1};
    \item les serveurs de \textbf{strate 2} sont généralement accessibles au public.

\end{enumerate}


\subsection{Configuration du serveur}
\label{subsec:configuration-serveur}

Voici les différentes étapes et options que nous avons effectuées :
\begin{itemize}

    \item[$\bullet$] activation des statistiques NTP;
    \item[$\bullet$] ajout de trois serveurs \textit{(un belge et deux européens)};
    \item[$\bullet$] activation de l'échange de l'heure avec tout le monde
    \textit{(aucune modification n'est acceptée)};
    \item[$\bullet$] activation de la synchronisation avec les machines du
    réseau local.

\end{itemize}


\subsection{Configuration du client}
\label{subsec:configuration-client}

Sur le client, la configuration est beaucoup plus simple :
\begin{itemize}

    \item[$\bullet$] activation des statistiques NTP;
    \item[$\bullet$] ajout du serveur local.

\end{itemize}
